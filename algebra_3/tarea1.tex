\documentclass{article}
\usepackage[utf8]{inputenc}
\usepackage{amsmath,amssymb}
\usepackage[left=2.5cm,right=2.75cm,bottom=3.5cm,top=2.5cm]{geometry}
\usepackage{enumerate}
\usepackage{amsthm}
\usepackage{tikz-cd}
\usetikzlibrary{cd}

\def\N{\mathbb{N}}
\def\Q{\mathbb{Q}}
\def\R{\mathbb{R}}
\def\Z{\mathbb{Z}}
\def\p[#1]{\sqrt{p_{#1}}}
\def\q[#1]{\sqrt{q_{#1}}}
\def\l[#1]{\lambda_{#1}}
% Esto sirve para abreviar comandos que uno usa con frecuencia y así tener que tipear menos. El comando para hacer la Q de los racionales es \mathbb{Q}; con esta línea, yo le digo a LaTeX que de ahora en más \Q va a significar eso mismo. %

\theoremstyle{definition}
\newtheorem{obs}{Observación}
\newtheorem*{demo}{Demotración}

\parskip 0.5em
% El espacio entre párrafos. %

\title{Ejercicio para entregar 1}
\author{Ismael Bejarano}
\date{}

\begin{document}

\maketitle

\subsection*{Ejercicio.} Sean $p_1, p_2, \dots, p_n \in \N$ primos distintos. 
Sea $E = \Q(\p[1],\p[2],\dots,\p[n])$.

\begin{enumerate}[(i)]
\item Probar que $[E : \Q] = 2^n$.
\item Sean $\l[1], \l[2], \dots, \l[n]$ números racionales no nulos. Probar
que los $2^n$ números de la forma 
$\pm\l[1]\p[1] \pm\l[2]\p[2] \pm \dots \pm\l[n]\p[n]$ son distintos dos a dos.
\item Sea $\alpha = \l[1]\p[1] + \l[2]\p[2] + \dots + \l[n]\p[n]$. Probar
que $E = \Q(\alpha)$.
\end{enumerate}

\subsection*{Solución:}

\begin{obs} 

Si $q \in \Q$  no es un cuadrado en $\Q$ entonces el polinomio 
$f(X) = X^2 - q$ es irreducible sobre $\Q$.

\end{obs}

\begin{demo}

Si fuera reducible existe $b \in \Q$ que es raíz de $f(X)$.
Pero entonces $-b$ también en solución. Luego
$f(X) = (X - b)(X + b)$, de donde $b^2 = q$ absurdo.

\end{demo}

\begin{obs}

Si $\alpha=\sqrt{q}$, $q \in Q$ no es un cuadrado y 
$F$ es una extensión sobre $\Q$ entonces 
$[F[\alpha]:F] \le [\Q(\alpha):\Q] = 2$.

\end{obs}

\begin{demo}

Tenemos el siguiente diagrama

\[
\begin{tikzcd}[column sep=tiny]
    & F(\alpha) \arrow[ld, dash] \arrow[rd, dash] & \\
\Q(\alpha) \arrow[rd, dash] & & F \arrow[ld, dash] \\
& \Q & 
\end{tikzcd}
\]

De donde inmediatamente tenemos que $[F(\alpha):F] \le [\Q(\alpha):\Q]$.
Como $[\Q(\alpha):\Q] = gr f(\alpha,\Q) = 2$ por la observación
anterior, entonces $[F(\alpha):F] \le 2$.

\end{demo}

\subsubsection*{(i)}

Vamos a ver algo más general si $q_1,q_2,\dots,q_n \in \Z$ no son cuadrados y
son primos entre sí dos a dos entonces $[\Q(\q[1],\q[2],\dots,\q[n]):\Q] = 2^n $.

Por inducción en $n$.

Caso $n=1$. Tenemos que $[\Q(\q[1]):\Q] = \textrm{gr}(f(\q[1], \Q)) = 2$.
Usando la primera observación.

Caso $n=2$. Tomamos $F = \Q(\q[1])$ y $E = F(\q[2])$, entonces
tenemos que $[E:\Q] = [E:F][F:\Q]$. Sabemos que $[F:\Q] = 2$, y que 
$[E:F] = 1$ ó $2$.

Supongamos que $[E:F] = 1$, en ese caso tenemos que $\q[2] \in F$.
Luego $a + b\q[1] = \q[2]$, donde $a,b \in \Q$.

Elevando al cuadrado y despejando tenemos que 

\[ a^2 + 2ab \q[1] + b^2 p_1 = \q[2] \]

Si $a=0$, entonces $b^2 \q[1] = \q[2]$. Multiplicando ambos lados 
por $\q[1]$ tenemos $b^2 q_1 = \sqrt{q_1 q_2} \in \Q$. O sea que
$q_2 r^4 = s^4 q_1$ es un cuadrado en

Tomando $b = \frac{r}{s}$ 
con $r,s \in \Z$ primos entre si.

Tenemos 

\[ r^2 p_1 = s^2 p_2 \]

de donde $p_1 \mid s$ (porque los $p_i$ son primos relativos entre sí). 
Similarmente $p_2 \mid r$. Si hacemos $s = p_1 s'$ y $r = p_2 r'$.

\[ r'^2 p_2 = s'^2 p_1 \]

Repitiendo el mismo proceso tenemos $r' = p_1 r"$ y $s' = p_2 s"$.
Pero entonces $r = p_2 p_1 r"$ y $s = p_1 p_2 s"$ lo que contradice
que $r$ y $s$ sean primos entre si.

Luego $a \ne 0$.

Si $b = 0$ entonces $a^2 = p_2$. Si tomamos $a = \frac{u}{v}$, con 
$u, v \in \Z$ primos entre si.

Tenemos 

\[ u^2 = p_2 v^2 \]

Luego tenemos $p_2 \mid u$, si hacemos $u = p_2 u'$ reemplazando
nos queda

\[ u'^2 p_2 = v^2 \]

Esto implica que $p_2 \mid v$, de donde $v = p_2 v'$, pero esto
implicaría que $u$ y $v$ no son coprimos. Absurdo.

Luego $b \ne 0$.

Como $ab \ne 0$ esto implica que $\sqrt{p_1} \in \Q$.
Absurdo.

Luego la suposición de que $[E:F] = 1$ es errónea y tiene
que ser $[E:F] = 2$.

O sea que $[\Q(\sqrt{p_1}, \sqrt{p_2}):\Q] = 2^2$

Caso $n > 2$. Supongamos que sea cierto para $n$ queremos probar que
es cierto para $n + 1$.

Sea $F = \Q(\sqrt{p_1},\dots,\sqrt{p_n})$ y sea $E = F(\sqrt{p_{n+1}})$.
Por hipótesis inductiva tenemos que $[F:\Q] = 2^n$, y además
sabemos que $[E:F] \le 2$.


Luego los $\sqrt{p_i}$ son algebraicos sobre $\Q$ y tenemos que:

\[\Q(\sqrt{p_1},\dots,\sqrt{p_n}) = \Q[\sqrt{p_1},\dots,\sqrt{p_n}] = \Q[\sqrt{p_1}]\dots[\sqrt{p_n}] \]

Además podemos ver que si $F = \Q(\sqrt{p_1},\dots,\sqrt{p_{n-1}})$ 
entonces tenemos el siguiente diagrama:

\[
\begin{tikzcd}[column sep=tiny]
    & F(\sqrt{p_n}) \arrow[ld, dash] \arrow[rd, dash] & \\
\Q(\sqrt{p_n}) \arrow[rd, dash, "2"'] & & \ \quad F\quad \arrow[ld, dash] \\
& \Q & 
\end{tikzcd}
\]

Por XXXX, tenemos que $[F(\sqrt{p_n}):F] \le [\Q(\sqrt{p_n}):\Q] = 2$.
Luego sólo puede ser que $[F(\sqrt{p_n}):F] = 1$ ó $2$.

Para probar que $[\Q(\sqrt{p_1},\dots,\sqrt{p_n}):\Q] = 2^n$ procedemos 
por inducción en $n$.

\end{document}
